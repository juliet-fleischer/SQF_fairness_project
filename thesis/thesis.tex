\documentclass[12pt]{article}

\usepackage{graphicx}
\usepackage{pgffor}
\usepackage{caption}
\usepackage{subfig}
\usepackage{subcaption}
\usepackage{tabularray}
\usepackage{color}
\usepackage{alltt}
\usepackage{float}
\usepackage{amsmath}
\usepackage{amsthm}
\usepackage{yhmath}
\usepackage{xcolor}
\usepackage{soul}
\usepackage{hyperref}
\usepackage{cleveref}
\usepackage{multirow}
\usepackage{pdfpages}
\usepackage{datetime}
\usepackage{makecell}
\usepackage{pdflscape}
\usepackage{array}
\usepackage{longtable}
\usepackage{booktabs}


\input{shortcuts.tex}



\newcommand{\minimage}[2]{%
  \includegraphics[width=0.08\textwidth]{../figs/cifar_images_examples/#1_#2.png}%
}

\newcommand{\minimagerow}[1]{
  \minimage{#1}{0} & \minimage{#1}{1} & \minimage{#1}{2} & \minimage{#1}{3} & \minimage{#1}{4} & \minimage{#1}{5} & \minimage{#1}{6} & \minimage{#1}{7} & \minimage{#1}{8} & \minimage{#1}{9}    
}

\newcommand*{\vertbar}{\rule[-1ex]{0.5pt}{2.5ex}}
\newcommand*{\horzbar}{\rule[.5ex]{2.5ex}{0.5pt}}


\newcommand{\mytitle}{FairML and the SQF Dataset}
\newcommand{\myname}{\large Juliet Fleischer}
\newcommand{\mysupervisor}{Dr. Ludwig Bothmann}

\newcommand{\thesuffix}[1]{%
  \ifnum#1=1 st%
  \else\ifnum#1=2 nd%
  \else\ifnum#1=3 rd%
  \else th%
  \fi\fi\fi}

% Manually define the date format with the ordinal suffix
\newcommand{\mydate}{%
  ~\ifcase\month\or
  January\or February\or March\or April\or May\or June\or July\or August\or
  September\or October\or November\or December\fi, \the\day\thesuffix{\day} \number\year}

\usepackage[a4paper, width = 160mm, top = 35mm, bottom = 30mm, 
bindingoffset = 0mm]{geometry}
\usepackage[utf8]{inputenc}
\usepackage{ragged2e}
\usepackage{xcolor}
% \usepackage[round, comma]{natbib}
\usepackage[backend=biber, style=authoryear]{biblatex}
\addbibresource{../literature.bib}
\addbibresource{../literature_manual.bib}
\usepackage{fancyhdr}
\newcommand{\changefont}{%
    \fontsize{8}{11}\selectfont
}
\hypersetup{
  colorlinks = true,
  linkcolor = black,
  urlcolor = black,
  citecolor = black}
\pagestyle{fancy}
\fancyhead{}
\fancyhead[R]{\changefont{\mytitle}}
\fancyfoot{}
\fancyfoot[R]{\thepage}
\setlength{\headheight}{14.5pt}
\setlength{\parindent}{0pt}
\interfootnotelinepenalty = 10000

% ------------------------------------------------------------------------------
% MAIN -------------------------------------------------------------------------
% ------------------------------------------------------------------------------
\IfFileExists{upquote.sty}{\usepackage{upquote}}{}
\begin{document}

% FRONT PAGE -------------------------------------------------------------------
 
\begin{titlepage}
\begin{center}
    
\LARGE
Seminar Thesis

\vspace{0.5cm}
      
\rule{\textwidth}{1.5pt}
\LARGE
\textbf{\mytitle}
\rule{\textwidth}{1.5pt}
   
\vspace{0.5cm}
      
\large
Department of Statistics \\
Ludwig-Maximilians-Universität München

\vfill

\Large
\textbf{\myname}

\vfill

\large

Munich, \mydate
      
\vfill

\includegraphics[width = 0.4\textwidth]{../figures/sigillum.png}

\vfill

\normalsize
Submitted in fulfillment of the requirements for the degree of B. Sc.
\\
Supervised by \mysupervisor

\end{center}
\end{titlepage}

% CONTENTS ---------------------------------------------------------------------

\pagenumbering{Roman}
\newpage
\begin{abstract}

With the increased presence of AI in our society topics of social justice and fairness have swept over to technical research fileds. In the first half of this paper we provide an introduction to the most common metrics and methods in fair machine learning. We then apply the theoretical concepts to the New York Stop, Question and Frisk dataset, which will showcase difficulties that come with fairness in practice.
This leads us to explore the problem of selection bias and how it affects algorithmic learning. For this, we turn our focus to studies that have worked with the SQF dataset and established interesting theoretical results: residual unfairness and bias reversal.
The main contribution of this paper lies in comparing and contrasting the different ways in which fairness has been studied for the Stop, Question, and Frisk dataset. We show that the challenges is not to identify the right approach, but rather to understand the implications and reasoning behind each method.

\end{abstract}


\newpage
\tableofcontents
\newpage


% CHAPTERS ---------------------------------------------------------------------

\pagenumbering{arabic}
    
\section{Introduction}
% \label{intro}
As algorithms are increasingly used to make decisions that affect people's lives, it is important to ensure that these decisions are fair. 
... the field of fiar machine learning. Interesting findings, recent developments, dynamic field.
limitations: unclear definitions, same datasets used over and over again.
This paper is an uncommon introduction to fairness. In "chapter ..." it provides the reader with the most common fairness metrics and methods, found in other introductions
but then in "Chapter ...." we apply them more uncommon dataset. SQF.
Since x the police in NYC is allowed to stop individuals if they have ...
Our case study will showcase difficulties that come with fairness in practice. This leads us to explore other studies that have worked with SQF data in "chapter ...". 

\newpage
\section{Related Work}
Given the abundance of fairness definitions there has been great efforts to summarise all of them. We do not claim a comprehensive overview of all fairness definitions in our paper but we mainly focus on definitions that become relevant in the case study or are necessary for a nuanced picture on fairness (give the reader some so they know they exist and people think about fairness also in this way).
We refer to \cite{verma2018}, \cite{castelnovo2022} for a good overview; \cite{corbett-davies} and \cite{barocas} for extensive and detailed discussions on fairness definitions and their limitations.
\cite{mehrabi2022} and \cite{caton2024} provide a great overview of fairness methods and sources of bias.
Related to SQF \cite{goel2016} stands out as they combine a socio-ecnonomic approach with advanced statistical methods. With their study, they support the claim that non-white individuals, are over-targeted by the New Yorker police. 

% use their study to back up my own choice for arrrestment as target and little racial disparities
\cite{Badr2022DTFANSP}

% introduce the interesting concept of bias reversal
\cite{RambachanBBOEFW}

% honestly, I want their problem setting and their estimation approach but nothin more
\cite{kallus2018}


% fairness addressed from a causal lense
In \cite{Khademi2019FADMELC} they address fairness from a causal perspective. They specifically measure a form of causal individual fairness for SQF and causal group fairness (FACE, FACT). But also here they come to different fairness statements depending on the metric.

% fairness (in SQF) really depends on the perspective you take! What task do train? What covariates do you take into account?
% How do you want to deploy the algorithm in the future? 
\newpage
\section{Fairness Metrics and Methods}
\label{sec:fairness_metrics_methods}
% \section*{Definitions of Fairness in Machine Learning}
When one starts to get into the topic of fairness in machine learning, it is easy to get overwhelmed by the sheer amount of definitions and metrics that are out there. In this chapter we try to group them in an intuitive way and motivate them in the hope to bring some clarity to readers. What all of them have in common is that they build on the idea of a protected attribute (PA) or alternatively called sensitive attribute. This is a feature present in the training data because of which individuals should not experience discrimination. Examples for sensitive attributes are race, sex and age. Often they are protected by law in some form. Coming back to differences in fairness definitions,  it is helpful to group fairness metrics in the following ways.
\begin{enumerate}
    \item Group fairness vs. individual fairness
    \item observational vs. causality-based criteria
\end{enumerate}

Broadly speaking, group fairness aims to create equality between groups and individual fairness aims to create equality between two individuals within a group. Group membership is encoded by the PA. Observational fairness metrics act descriptive and use the observed distribution of random variables characterizing the population of interest to assess fairness while causality-based criteria make assumptions about the causal structure of the data and base their notion of fairness on these structures.
On the basis of these fundamental ideas, a plethora of formalizations have emerged. Most of them concern themselves with defining fairness for a binary classification task and one PA. For this work, we will also stay within this setting. For the subsequent sections let $Y \in \{0, 1\}$ be the true label, $\hat{Y} \in \{0, 1\}$ be the prediction label, $\hat{R}$ bet the prediction score, $A$ be the sensitive attribute and $X$ encode the non-sensitive attributes.

\subsection{Group fairness}
\begin{table}
    \centering
    \begin{tabular}{lll}
        \toprule
        Independence & Separation & {Sufficiency} \\
        \midrule
        $\hat{Y} \perp A$ & $\hat{Y} \perp A | Y$ & {$Y \perp A | \hat{Y}$}\\
        \bottomrule
    \end{tabular}
    \caption{Group fairness metrics}
    \label{tab:group_fairness}
\end{table}

The groups metrics presented in the following are observational metrics. They can be separated into three main categories shown in \autoref{tab:group_fairness}, depending on which information they use.

\subsubsection*{Independence}
Independence is in a sense the simplest group fairness metric. It requires that the prediction $\hat{Y}$ is independent of the protected attribute $A$. This is fulfilled when for each group the same proportion is classified as positive by the algorithm. In other words, the positive prediction ratio (ppr) should be the same for all values of $A$. For a binary classification task with binary sensitive attribute this can be formalized as \\
\textbf{demographic parity/statistical parity}
$$P(\hat{Y} | A = a) = P(\hat{Y} | A = b)$$
Conditional statistical parity is an extention of this as it allows to condition on $A$ and a set of legitimate features $E$. In the context of SQF, predictive parity would mean that we require equal prediction ratios between PoC and white people while conditional statistical parity requires equal prediction ratios between PoC and white people who \textit{live within the same borough} of New York $(E = borough)$. This can be seen as a more nuanced approach, as it allows tacking additional information into account.
The other two categories of group fairness metrics can both be derived from the error matrix.

\subsubsection*{Separation}
Separation requires independence between $\hat{Y}$ and $A$ conditioned on the true label $Y$. This means that the focus is on equal error rates between groups, which gives rise to the following list of fairness metrics:
\begin{itemize}
    \item Equal opportunity/ False negative error rate balance $$P(\hat{Y} = 0 | Y = 1, A = a) = P(\hat{Y} = 0 | Y = 1, A = b)$$
    \item Predictive equality/ False positive error rate balance $$P(\hat{Y} = 1 | Y = 0, A = a) = P(\hat{Y} = 1 | Y = 0, A = b)$$
    \item Equalized odds $$P(\hat{Y} = 1 | Y = y, A = a) = P(\hat{Y} = 1 | Y = y, A = b) \forall y \in \{0, 1\}$$ 
    \item Overall accuracy equality: $$P(\hat{Y} = Y | A = a) = P(\hat{Y} = Y | A = b)$$ 
    \item Treatment equality: $$\frac{\text{FN}}{\text{FP}} \big|_{A = a} = \frac{\text{FN}}{\text{FP}} \big|_{A = b}$$
\end{itemize}

Equal opportunity requires the false negative rates, the ratio of actual positive people that were wrongly predicted as negative, to be equal between groups.
Therefore, it is also called false negative error rate balance. When there false negative rates are equal between groups, then the true positive rates between groups are also equal. Thus to formulate equal opportunity one can equivalently require equal trupe positive rates between groups.
Predictive equality follows the same principle as equal opportunity but instead of focusing on the false negatives, it focuses on the false positives. Again, if a classifier has equal false positive rates between groups, it also has equal true negative rates.
Equalized odds combines equal opportunity and predictive equality. It requires that the false positive and true positive rates are equal between groups, and is in this sense stricter than either of them alone. \\
In itself, these error rates are detached from the context of fairness and used in general in machine learning to assess the performance of a classifier. In essence the group metrics we outlined so far do nothing other than picking a performance metrics from the confusion matrix and requiring it to be equal between two (or more) groups in the population.
This means the well-known trade-offs for example between false positive and true positive rate are also present in the fairness metrics. As more people get correctly classified as positive usually also more people get wrongly classified as positive. {\color{red}Source}
With this comes the difficulty to choose "the right" metric for the specific task. In general one can think about this in the same way as when choosing a performance metric for a binary classifier.
In setting in which a positive prediction leads to a harmful outcome, as in the SQF setting, it often makes sense to focus on minimizing the false positive rate, while a higher false negative rate is accepted as a trade-off.
This argumentation follows the idea of. The authors distinguish between punitive and assistive tasks to help choose the right fairness metric. For punitive tasks metrics that focus on false positives, such as predictive equality are more relevant. For assistive tasks, such as deciding who receives some kind of welfare, a focus on minimizing the false negative rate could be more relevant, so equal opportunity would be more suitable. We note that there is dedicated work that assists in finding the right fairness metric for a given situation and refer to \cite{makhlouf2021} for an alternative approach and more depth.

\subsubsection*{Sufficiency}
Sufficiency requires independence between $Y$ and $A$ conditioned on $\hat{Y}$. Intuitively this means that we want a prediction to be equally credible between groups. When a white person gets a positive prediction the probability that it is correct should be they same as for a black person. This leads to the following fairness metrics:
\begin{itemize}
    \item Predictive parity/ outcome test requires that the probability of actually being positive, given a positive prediction is the same between groups. $$P(Y = 1 | \hat{Y} = 1, A = a) = P(Y = 1 | \hat{Y} = 1, A = b)$$
    \item Equal true negative rate follows the same principle as predictive parity. It requires that the probability of actually being negative, given a negative prediction is the same between groups.: $$P(Y = 0 | \hat{Y} = 0, A = a) = P(Y = 0 | \hat{Y} = 0, A = b)$$
    \item If we instead look at errors again, we can require equal false omission rates: $$P(Y = 1 | \hat{Y} = 0, A = a) = P(Y = 1 | \hat{Y} = 0, A = b)$$
    \item Or equal false discovery rate: $$P(Y = 0 | \hat{Y} = 1, A = a) = P(Y = 0 | \hat{Y} = 1, A = b)$$
\end{itemize}

Just a for the \textit{Separation} metrics one can combine two of these  \textit{Sufficiency} metrics and require them to hold simultaneously to get a stricter requirement.
While it is easy to get lost by the amount of fairness definitions in the beginning, taking a closer look, it becomes clear that they are constructed in a structured way. In fact, equal false omission rate and equal false discovery rate were not introduced in the paper \cite{verma2018} but are implemented in \texttt{mlr3fairness}, and evidently follow the same pattern as the other metrics.

\begin{center}
    \renewcommand{\arraystretch}{1.5}  % Increase row height for a more square-like appearance
    \begin{tabular}{c|c|c|}
        \hline
        & \(Y = 0\) & \(Y = 1\) \\
        \hline
        \(\hat{Y} = 0\) & TN & FN \\
        \hline
        \(\hat{Y} = 1\) & FP & TP \\
    \end{tabular}
    \captionof{table}{Confusion matrix}
    \label{tab:confusion_matrix}
\end{center}

\subsubsection*{Score-based fairness metrics}
Most (binary) classifiers work with predictions scores and a hard label classifier is applied only afterwards in form of a threshold criterion. It should therefore come as no surprise that instead of formulating fairness with $\hat{Y}$ there exist fairness metrics that use the score $S$, which typically represents the probability of belonging to the positive class. Instead of conditioning on $\hat{Y}$ as Separation metrics, we can simply condition on $S$ and define Calibration:
$$P(Y = 1 | S = s, A = a) = P(Y = 1 | S = s, A = b)$$
Calibration requires that the probability for actually being positive, given a score $s$ is the same between groups. So the idea is a more fine-grained version of predictive parity. As the score can usually take values from the whole real number line, this can in practice be implemented by binning the scores. See \cite{verma2018} for an example.

\subsection{Individual fairness}
If we want to equalize e.g. the false positive rates between two groups and currently group $a$ has a higher false positive rate than group $b$, this would lead us to lowering the prediction threshold for b, such that more actual negative people would get classified as positive. Or if we would need to set a higher threshold for group $a$, such that it becomes harder for them to be classified as positive. Depending on the context, either option can seem unfair.By trying to equalize a given metric between groups, it can happen that individuals within a group are treated unequally. Individual metrics therefore shift the focus. The underlying idea of fairness is that similar individuals should be treated similarly.

\subsubsection*{Fairness through awareness (FTA)}
FTA formalizes this idea as Lipschitz criterion. $$d_Y(\hat{y_i}, \hat{y_j}) \leq \lambda {d_X}(x_i, x_j)$$
$d_Y$ is a distance metric in the prediction space, $d_X$ is a distance metric in the feature space and $\lambda$ is a constant.
The criterion puts an upper bound to the distance between predictions of two individuals, which depends on the features of them. In other words, if two people are close in the feature space, they also should be close in the prediction space. The challenge of FTA is the definition of the equality in the feature space \cite{castelnovo2022}.
In the SQF context, it could make sense to define similar individuals based on yearly income, age and neighbourhood.
Yet one could easily argue that taking the criminal history into account is important as well. After the decision for a legitimate set of features has been made, the next challenge is to choose a distance metric that appropriately captures the conceptual definition of similarity defined via the selected features.
FTA does not have one clear solution and requires domain knowledge and the choice of $d_X$ should take context-specific information into account.

\subsubsection*{Fairness through unawareness (FTU) or blinding}
In contrast to FTA, blinding should give a simple, context-independent rule. It tells us to not use the protected attribute explicitly in the decision-making process. When training a classifier this means discarding the PA during training.
Since FTU is a more procedural rule than a mathematical definition, there exist multiple ways to test whether the blinding worked for a classifier. One approach is to simulate a doppelgänger for each observation in the dataset. This doppelgänger has the exact same features except the protected attribute, which is flipped.
If both these instances have the same prediction, the algorithm would satisfy FTU \cite{verma2018}. \footnote{This can be seen as a from of FTA, in which we chose the distance metric to measure a distance of zero only if two people are the same on all their features except for the protected attribute. In this special case FTA and FTU are measured in the same way.} Other ways to assess FTU can be found in \cite{verma2018}. 
A problem blinding has been proxies. These are variables that are strongly correlated with the sensitive attribute. It is not enough to simply mask the information of the sensitive attribute during training because discrimination can persist via these proxies.
For SQF this would mean that we remove the race attribute during training.
A person's ethnicity, however, is strongly correlated with their place of residence. Thus, indirect discrimination based on ethnicity remains, even though the information was not directly available during training. \textbf{Suppression} extends the idea of blinding and the goal is to develop a model that is blind to not only the sensitive attribute but also the proxies. The drawback is, that it is unclear when a feature is sufficiently high correlated with the sensitive attribute to be counted as proxy. Additionally, we could lose important information by removing too many features \cite{castelnovo2022}.

\subsection{Causality-based fairness metrics}
In contrast to observational fairness metrics, causality-based notions ask whether the sensitive attribute was the \textit{reason} for the decision. If a certain (harmful) decision was made \textit{because of} the value of the sensitive attribute of a person, we deem the algorithm as unfair.
% There are causality-based concepts that focus on group-level fairness and also some that focus on individual-level fairness. We want to give an intoduction to all of them, but since this category requires a new theory we will not get into great detail.

\textbf{Group-level}: FACE, FACT (on average or on conditional average level) \parencite{Zafar2017PPNFC}\\
\textbf{Individual-level}: counterfactual fairness, path-based fairness \parencite{kusner} 
The two most common individual fairness metrics are counterfactual fairness and path-based fairness.


\subsection{Comparison and Summary}

The difference between observational and causal clear, really different approach. The division in group and individual fairness metric actually more of a nuanced differentiation. The observational metrics can rather be ordered on a plane, depending on how much information of the situation via other features X they allow.
Traditional group metrics like demographic parity, equal error rate metrics and sufficiency metrics only work with the distribution of $Y, \hat{Y}, X, A$. The individual fairness metrics take more information of the non-sensitive feature into account in order to define similarity. Metrics such as conditional demographic parity lie in between, as we allow for a relevant subset of non. Sensitive feature to be part of the definition.
\cite{castelnovo2022} therefore depict this as a plane.
The amount of approaches to measure fairness shows the complexity of the topic. There is not \textit{the} right fairness metric to choose, but there can be the best one depending on the context and the data. The next section will present ways to digitate algorithmic bias once detected by one of the fairness metrics.

To compare the group fairness criteria, sufficiency takes the perspective of the decision-making instance, as usually only the prediction is known to them in the moment of decision. For example, the police, who do not yet know the true label at the time when they are supposed to decide whether someone would become a criminal.
As separation criteria condition on the true label Y it is suitable when we can be sure that $Y$ is free from any bias, so to say when $Y$ was generated via an objectively true process (this will become clearer in the chapter on bias).
Independence is best, when we want to enforce a form of equality between groups, regardless of context or any potential personal merit. While this seems to be useful in cases in which the data contains complex bias, it is unclear whether these enforcements have the intended benefits, especially over the long term. {\color{red}{Reference?}}.
It is good to understand the difference in perspectives each of the group fairness metrics take, because many of them cannot be satisfied simultaneously. This is known as the impossibility theorem \cite{hardt2016}. This means one has to decide on either Independence, Separation or Sufficiency and the choice should fit the context of the data and the decision-making process. Lastly, we note that these are not all the group fairness metrics that exist, but broadly speaking other metrics are variations of the presented ones. Some more metrics are listed in the appendix.



\subsection{Fairness methods}
After defining fairness in a mathematical sense, the question arises how a classifier can be modified to satisfy the chosen definition of fairness. This is what fairness methods deal with.
Depending on their position in the machine learning pipeline, we distinguish between
\begin{enumerate}
    \item Pre-processing methods
    \item In-processing methods
    \item Post-processing methods
\end{enumerate}
Pre-processing methods follow the idea that the data should be modified before training, so that the algorithm learns on "corrected" data. Reweighing observations before training is an example for a preprocessing method. The idea is to assign different weights to the observations based on relative frequencies, so that the algorithm learns on a balanced dataset (\cite{caton2024}).\\
In-Processing methods modify the optimization criterion, such that it also accounts for a chosen fairness metric. Introducing a regularization term to the loss function is one example of such modifications.\\
Post-processing methods work with black box algorithms, just like preprocessing methods. We only need the predictions from the model to adjust them so that again a chosen fairness metric is fulfilled. One example for this is thresholding, where we set group specific thresholds to re-classify the data after training (\cite{hardt2016}).
Depending on the task (regression, classification) and the model there are highly specified and advanced methods. For the case study in chapter 3, we limit ourselves to methods implemented in the \texttt{mlr3fairness} package.
% mlr3fairness currently has two preprocessing methods, one postprocessing method and several fairness adjusted models implemented. We decide to use a reweighing methods that works with assigning weights to the observations to equalise the distribution of $P(Y|PA)$.
% The inprocessing method is a fairness-adjusted logistic regression implemented in mlr3fairness inspired by Zafar et. al. This method optimises for statistical parity (independence). The postprocessing method we choose aims for equalised odds and it works by randomly flipping a subset of predictions with pre-computed probabilities in order to satisfy equalised odds constraints.

\subsection{Bias and the feedback loop}
Before applying the theory to real-world data, it remains to introduce different types of biases and the context in which a machine learning model is usually embedded.
Deployed as an ADM, the model assists in decisions such as whether someone gets admitted to college, receives a loan or is released from prison. It thereby indirectly contributes to shaping our reality.\\
\cite{mehrabi2022} conceptualise the situation in form of the \textit{data, algorithm, and user interaction feedback loop} (\autoref{fig:bias_loop}), which can be understood as follows.
We as a society make decisions, which are reflected in our reality. The reality is made measurable by collecting data. The algorithm learns from this data to make an optimal prediction, on which the decision-maker bases their judgement. The new choice will shape our reality again, which reflects in updated data.\\ 
At each stage, bias can be introduced into the process. More dangerous, bias can even be amplified as the algorithm influences decision-making on a large scale.
Consequently, every fairness project comes with the responsibility to understand the data-generating process and gain clarity on how the algorithm will be deployed in the real world.\\
Moreover, the \textit{data, algorithm, and user interaction feedback loop} helps clarify which type of bias might be relevant in a given situation by placing it at a specific position in the feedback loop. Distinguishing between bias mechanisms can be crucial and should influence the definition of fairness and the choice of fairness adjustments in a given situation. This will also become evident in the following section where we examine the SQF dataset.

\begin{figure}
    \centering
    \includegraphics[width=0.7\textwidth]{../figures/bias_loop.png}
    \caption{The \textit{data, algorithm, and user interaction feedback loop} as described by \cite{mehrabi2022}. Different categories of bias can be introduced at each stage of the process.}
    \label{fig:bias_loop}
\end{figure}

% \subsection*{Bias}
% We want to end this general introduction into fair machine learning by outlining the context in which the algorithm is usually embedded. On this note we also advice practitioners to think about the source of bias that could be present in your situation, as this \textit{should} influence how fairness is defined and what fairness adjustments are appropriate. This will motivate the potential difficulties that can arise when implementing fairness in the real world.
% \cite{caton2024} describe the situation as follows. The algorithm is embedded in a feedback loop with the user and data.
% We as a society make decision, which reflect our reality. We make our reality measurable by collecting data. The algorithm learns from this data and makes predictions, on which we base new decisions. 
% At each of these three points bias can be introduced into the process and, above all, bias can also be reinforced in the course of this process.
% In the context of the Stop, Question, and Frisk data, historical bias and selection bias are probably the most relevant sources of bias.
% Historical bias can shows itself in different ways. In our case it would mean that we assume that some people in our data have repeatedly experienced discrimination in terms of being arrested.
% Selection bias refers to the fact that the data is not representative of the population of New York City, because the decision to stop someone is based on a biased decision policy.




\newpage
\section{Case Study: Stop, Question, and Frisk}
\label{sec:case_study}
% \subsection*{Fairness Experiment: Stop, Question, and Frisk data}
After introducing the theoretical tools for assessing fairness, we turn to a case study on the stop-and-frisk practice. A police officer is allowed to stop a person if they have reasonable suspicion that the person has committed, is committing, or is about to commit a crime.
During the stop the officer is allowed to frisk a person (pat-down the person's outer clothing) or search them more carefully.
The stop can result in a summon, an arrest or no further consequences. After a stop was made, the officer is required to fill out a form, documenting the stop. This data is published yearly by the NYPD.
As mentioned in the introduction the so-called "New York strategy" \cite{gelman2007} is highly controversial. The aggressive way in whcih the stop-and-frisk practice was being implemented during 2004 to 2012 in NYC was indeed deemed unconstitutional in 2013, violating the fourth and fourteenth amendment {\color{red} Source}


\subsection{Fairness Experiment: Setup}
For our analysis the task is to predict the arrest of a suspect. We compare the following models in terms of fairness and model performance, measured by the difference in true positive rates and the classification accuracy respectively.:
\begin{itemize}
    \item Regular Random Forest
    \item Reweighing to balance disparate impact metric (Pre-Processing)
    \item Classification Fair Logistic Regression With Covariance Constraints Learner (In-Processing)
    \item Equalized Odds Debiasing (Post-Processing)
\end{itemize}
More details about the methods can be found in the \texttt{mlr3} documentation \cite{mlr3_book}.  
For reweighing, see \href{https://mlr3fairness.mlr-org.com/reference/mlr_pipeops_reweighing.html}{mlr3fairness Reweighing}.  
For fair logistic regression, refer to \href{https://rdrr.io/cran/mlr3fairness/man/mlr_learners_classif.fairzlrm.html}{Fair Logistic Regression}.  
For equalized odds, check \href{https://mlr3fairness.mlr-org.com/reference/mlr_pipeops_equalized_odds.html}{Equalized Odds}.  

% Reweighing: https://mlr3fairness.mlr-org.com/reference/mlr_pipeops_reweighing.html
% Fair logistic regression: https://rdrr.io/cran/mlr3fairness/man/mlr_learners_classif.fairzlrm.html
% EOd: https://mlr3fairness.mlr-org.com/reference/mlr_pipeops_equalized_odds.html
% mlr3book: https://mlr3book.mlr-org.com/chapters/chapter14/algorithmic_fairness.html
% https://mlr3fairness.mlr-org.com/#debiasing-methods

\subsection{Data description}
\begin{figure}
  \centering
  \begin{minipage}{0.49\textwidth}
      \centering
      \includegraphics[width=\textwidth]{../figures/sqf_case_study_plot6.pdf}
  \end{minipage}
  \hfill
  \begin{minipage}{0.49\textwidth}
      \centering
      \includegraphics[width=\textwidth]{../figures/sqf_case_study_plot14.pdf}
  \end{minipage}
  \caption{Bar plot comparing the distribution of ethnic groups across boroughs in the SQF 2023 and NYC from 2020 Census (left). On the right a comparison of the estimated borough-wise crime rate per 100,000 citizens with the ethnic distribution of SQF stops.}
  \label{fig:race_distributions}
\end{figure}

As they were the most recent at the time of writing this paper, we work with the stops from 2023. The raw 2023 dataset consists of 16971 observations and 82 variables. We first discarded all the variables that have more than 20\% missing values, which leaves 34 variables.
From this reduced dataset we filter out the complete cases and end up with 12039 observations. \footnote{Simply discarding the missing values and only training on complete cases is discouraged by \cite{fernando2021}. We opt for this approach regardless, since imputation of the missing values is not straight forward
but treating missing values as an extra category will introduce complications when we implement fairness methods.} \\

We summarize "Black Hispanic" and "Black" into the group "Black" and  "American Indian/ Native American" and "Middle Eastern/ Southwest Asian" into the "Other" category. Black people are by far most often stopped, making up 70\% of the total stops; yet, according to 2020 census data black people make up only 20\% of the city's population \autoref{fig:race_distributions}. At the same time white people form the majority of New York citizens (30\%) but contribute with only 6\% to the stops. 
After 2012 there has been a stark decline in stops and the police is known to focus their attention on high crime areas. Therefore, we further look at each borough. 
The most stops in 2023 occur in Bronx and Brooklyn. Based on report of the NYPD and population statistics from 2020, the Bronx also has the highest estimated crime rate per 100,000 citizens. Manhattan is not far behind in crime rate, but has fewer stops. Note that Bronx and Brooklyn happen to be the boroughs with the highest proportion of black citizens \autoref{fig:race_distributions}.\\

  
Given the historical context of stop-and-frisk, the question arises if a classifier trained on data from the unconstitutional period will perform differently.
We choose data from 2011 as it is the year with the most stops. We carry out the same data cleaning steps for the 2011 data as before, starting with 685724 recorded stops and reducing this to 651567 clean observations. Note that these are more than 50 times more stops than in 2023.
The 2011 data has substantially more low-risk stops, only around 6\% of stops result in an arrest. This is a stark contrast to the 31\% in 2023. In the data, the differences in arrestment rate between groups are slightly lower for 2011 and the highest arrestment rate remains to be for the white group.\\

As features, we select variables that should resemble the information that were available to the officer at the time they made the decision to arrest the person. This includes information about the development of the stop, e.g. whether the person was frisked or a summon issued. We assume that all of these constitute "smaller" hits that happen before an officer chooses the most extreme consequence, an arrest. Additionally, we control for factors, such as the time of the stop or whether the officer was wearing a uniform. This selection of features is inspired by \cite{Badr2022DTFANSP}.

\subsection{Results of the Fairness Experiment}
For the training of the classifiers, we dichotomize the race attribute by grouping "Black" and "Hispanic" as people of colour ("PoC") and "White", "Asian", and "Other" as white ("White"). We run a five-fold cross validation and show the results in \autoref{fig:fairness_experiment}. In the bottom right corner we find fair and accurate classifiers. In terms of fairness reweighing and the equalized odds post-processing method perform best. However, the regular random forest classifier comes close to their fairness performance and performs slightly more accurate. Somewhat surprisingly, it does not make any difference for the fairness if the classifier is trained on 2011 or 2023 data.
We examined the model closer and find that due to the low prevalence in the population, a classifier trained on 2011 data primarily suffers from the highly skewed distribution of arrests. The classifier largely predicts the negative label for \textit{anyone} regardless of race, which overshadows potential fairness concerns. The fairness adjusted logistic regression performs worst in terms of accuracy and fairness.
As the picture could change depending on the chosen fairness metric (y-axis), we also tried out other metrics, such as equalised odds or predictive parity. In all cases the regular random forest does not perform worse in terms of fairness but better in terms of accuracy than most fairness adjusted classifiers.
% result plot fairness experiment
\begin{figure}
    \centering
    \includegraphics[width=0.7\textwidth]{../figures/sqf_case_study_plot3.pdf}
    \caption{Comparison of learners with respect to classification accuracy (x-axis) and equal opportunity (y-axis) across (dots) and aggregated over (crosses) five folds.}
    \label{fig:fairness_experiment}
\end{figure}

Since the classifiers perform similarly, we choose the regular random forest trained on 2023 to examine the model closer.
On the left we plot the prediction score densities for each group in \autoref{fig:fairness_density}. We can see that in general white people tend to have higher predicted probabilities than PoC. The mode for the scores for non-white individuals is around 0.05 while it is around 0.125 for white individuals. The score resembles the probability of being predicted positive (arrested).
On the right \autoref{fig:fairness_density} we plot the absolute difference in selected group fairness metrics.
Exact equality of the group metrics cannot be expected in practice, so it is common to allow for a margin of error $\epsilon$. Taking $\epsilon = 0.05$, the classifier is fair according to each of the selected metrics, though the difference in positive predictive rates is close to 0.05.
For a more nuanced picture, we additionally report the group-wise error metrics in \autoref{tab:groupwise_metrics_2023}.
The true positive rate, false positive rate, and the accuracy is basically identical between the two groups. So the Separation metrics are fulfilled. More ore less notable differences can only be seen in the Sufficiency metrics: the negative predictive values/ positive predictive value.\\

\begin{figure}
    \centering
    \begin{minipage}{0.49\textwidth}
        \centering
        \includegraphics[width=\textwidth]{../figures/sqf_case_study_plot1.pdf}
    \end{minipage}
    \hfill
    \begin{minipage}{0.49\textwidth}
        \centering
        \includegraphics[width=\textwidth]{../figures/sqf_case_study_plot2.pdf}
    \end{minipage}
    \caption{Fairness prediction density plot (left) showing the density of predictions for the positive class split by "PoC" and "White" individuals. The metrics comparison barplot (right) displays the model's absolute differences across the specified metrics.}
    \label{fig:fairness_density}
\end{figure}

% results table
\begin{table}[ht]
  \centering
  \begin{tabular}{rrrrrrr}
    \hline
   & tpr & npv & fpr & ppv & fdr & acc \\ 
    \hline
    PoC & 0.75 & 0.89 & 0.07 & 0.84 & 0.16 & 0.88 \\ 
    White & 0.74 & 0.85 & 0.06 & 0.89 & 0.11 & 0.86 \\ 
     \hline
  \end{tabular}
  \caption{Groupwise Fairness Metrics (2023)} 
  \label{tab:groupwise_metrics_2023}
\end{table}

All in all, it seems like a classifier trained on SQF data to predict the arrest of a suspect is not discriminatory against PoC. In contrast, it even performs better on many of the common performance metrics for PoC than for white people. \cite{Badr2022DTFANSP} have similar findings.\\
In their study they choose six representative machine learning algorithms (Logistic Regression, Random Forest, Extreme Gradient Boost, Gaussian Naïve Bayes, Support Vector Classifier) to predict the arrest of a suspect. Fairness is measured with six different metrics (Balanced Accuracy, Statistical Parity, Equal Opportunity, Disparate Impact, Avg. Odds Difference, Theil Index) and separate analysis are conducted with sex and race as PA.
They compare the fairness of the regular learner to the fairness of learner with a pre-processing method (reweighing) and a post-processing method (Reject Option-based Classifier). All in all, they find that the regular models to not perform worse in terms of fairness than the fairness adjusted models. This leads them to conclude "[...] that there is no-to-less racial bias that is present in the NYPD Stop-and-Frisk dataset concerning colored and Hispanic individuals."
What both of our case studies have in common is that the models were trained on recent data. We trained our model on 2023 stops and \cite{Badr2022DTFANSP} used 2019 stops. Since the judgement of how stop-and-frisk was implemented in NYC in 2013, the number of stops has decreased significantly and citizens are in generally less often stopped. After 2014 the stops have been consistently kept at a low level. See this website for a visualization and information on the governing police administration at a given period \href{https://www.nyclu.org/data/stop-and-frisk-data}{stop-and-frisk over time}.
\cite{Badr2022DTFANSP} see this as explanation for their results and state "The NYPD has taken crucial steps over the past years and significantly reduced racial and genderbased bias in the stops leading to arrests. This conclusion nullifies the common belief that the NYPD Stop-and-Frisk program is biased toward colored and Hispanic individuals." Is this the whole picture?





\newpage

\section{Studies on the SQF Dataset}
\label{sec:studies}
\subsection{Approaches to fairness in SQF}

Before going into detail about a specific study, we provide a tabular overview of the different approaches to fairness in the SQF data. We will go into more depth into two of them in the following.

\begin{table}[h]
    \centering
        \begin{tabular}{|m{2cm}|m{3cm}|m{2.5cm}|m{3.5cm}|m{2.8cm}|}
            \hline
            \textbf{Authors} & \textbf{Task} & \textbf{Model} & \textbf{Fairness Metric} & \textbf{Results} \\
            \hline
            \cite{kallus2018} 
            & Predict prob. of innocence (no weapon) 
            & Log. Regression 
            & Equal Opportunity, Equalized Odds 
            & Bias against PoC \\ 
            \hline
            \cite{rambachan2016} 
            & Possession of contraband
            & Log. Regression 
            & No explicit fairness metric; evaluate prediction function properties 
            & No bias against PoC\\
            \hline
            \cite{Badr2022DTFANSP} 
            & Predict probability of arrest 
            & Log. Regression, RF, XGBoost, GNB, SVC 
            & Balanced Accuracy, Stat. Parity, Equal Opportunity, Disparate Impact, Theil Index 
            & No bias against PoC \\ 
            \hline
            \cite{Khademi2019FADMELC} 
            & Predict probability of arrest 
            & Weighted regression models
            & FACE causal fairness (group), FACT fairness (individual) 
            & No group bias, but individual bias \\ 
            \hline
            \cite{goel2016} 
            & Predict possession of weapon 
            & (Penalized) Log. Regression 
            & No explicit fairness metric; group-wise hit rates 
            & Bias against Black and Hispanic\\ 
            \hline
        \end{tabular}
        \caption{Overview of SQF-related fairness studies. This table summarizes findings from five key studies evaluating the fairness of the stop-and-frisk policy. Depending on the task and model used, the studies reach different conclusions.}
        \label{tab:sqf_summary}
\end{table}

% \subsection{Sources of bias in the SQF data}
One of the main challenges with the NYPD's data is that, when evaluating the fairness of stop-and-frisk as a policing strategy, various tasks can be formulated to address the question. Only some of them are suitable to make conclusions about the fairness of the stop-and-frisk policy as a whole.\par
Like \cite{Badr2022DTFANSP}, we trained a classifier to predict arrests and used group metrics to assess fairness. However, given that we also trained a "fair" classifier on data from 2011, but the stop-and-frisk practice was officially declared unconstitutional for 2004 to 2012, the task we (and \cite{Badr2022DTFANSP}) chose, is not a reliable indicator of the overall fairness of the policy.\par
To answer the question of fairness in stop-and-frisk other studies take a step back and identify a problem with how the data is generated. They formalize and acknowledge that the discrimination in SQF does not solely lie in the outcome of the stop but the decision to stop someone in the first place.


\subsection{Residual unfairness}

In their paper \textbf{Residual Unfairness in Fair Machine Learning from Prejudice Data} \cite{kallus2018} conceptualize the problem as shown in \autoref{fig:selection_bias}.
A person is defined by their sensitive feature $(A)$ and non-sensitive features $(X)$. For each person in the population of interest a police officer decides whether to stop them $(Z = 1)$ or not $(Z = 0)$. This mechanism can be seen as a category of selection bias and, referring back to the feedback loop (\autoref{fig:biasLoop}), is introduced by the user.\par
In the SQF context, we can imagine that the police are generally more suspicious towards PoC than white people, resulting in more stops of the former. Alternatively, one could argue that they are more likely to stop individuals in high-crime areas, which happen to be mostly low-income neighbourhoods largely populated by PoC.\par
Naturally, we can only know the outcome $Y \in \{0, 1\}$ of a stop for the individuals who were stopped. This can create a situation in which the training data produced by the biased decision policy is not be representative for the population the algorithm will be deployed on.
\cite{kallus2018} distinguish between target population and training population in such scenarios. The target population is the one on which we want to use the ADM on while the training population are the observations the biased decision policy chose to include in the sample and on which the algorithm is trained.\par
The problem for fairness in this case is that fairness adjustments of the learner (trained on the biased sample) do not translate to its deployment on the target population. Even fairness-adjusted classifiers can discriminate against the same group that has historically faced discrimination (\cite{kallus2018}). They call the remaining disparities in fairness metrics after fairness adjustments have been made \textbf{residual unfairness}.\par
% The indicator $T \in \{0, 1\}$ tell us whether a person belongs to the target population. If $T = 1$ constantly, it means that the algorithm should be deployed for the entire population of NYC.\\
At this point, we refer back to \autoref{fig:race_distributions}. The left plot shows a clear difference between the racial distribution in the SQF data and the city as a whole. In terms of race, the sample is clearly not representative for NYC\footnote{It can be questioned whether it makes sense to require the SQF sample to be representative for the population of NYC. It might make more sense to require that it is representative of the population of \textit{criminals} in NYC.}. At the same time the estimated borough-specific crime rates also differ from the distribution of stops per borough as seen in the right plot.\par
\cite{kallus2018} demonstrate that their theoretical findings are reflected in the SQF data. Their task is to predict a person's innocence. They define innocence as not carrying an illegal weapon and guilt as carrying one. The reasoning behind this approach is that the discriminated group is the one more frequently falsely accused of possessing an illegal weapon. The authors find that non-white individuals are indeed more often wrongfully classified as guilty. Even after applying a post-processing strategy to achieve equalized odds, unfairness against PoC persists when the classifier is used on NYC’s overall target population.
\newpage
\begin{figure}[htpb]
    \includegraphics[width=0.5\textwidth]{../figures/selection_bias.png}
    \caption{Selection bias in the SQF data, as conceptualized by \cite{kallus2018}. The true label is only known for the stopped individuals $(Z = 1)$.}
    \label{fig:selection_bias}
\end{figure}

\subsection{Bias in, bias out?}

Another perspective is offered by \cite{rambachan2016} in their paper \textbf{Bias In, Bias Out? Evaluating the Folk Wisdom}.
While the main message of \cite{kallus2018} is that even fairness adjusted classifiers exhibit the "bias in, bias out" mechanism \cite{rambachan2016} argue that it depends on the chosen classification task.

Similar to \cite{kallus2018} they are interested in whether a person carries a contraband $Y \in \{0, 1\}$. The paper assumes the police is a taste-based classifier against African-Americans. This means they hold some form of prejudice against the group of African-Americans that influences their decision to stop a member of this group. More precisely, they see the biased-decision policy in the decision to \textit{search} someone.\par
When we previously defined the biased selection mechanism as $Z=1$ corresponding to stop and $Z=0$ corresponding to no stop, this study sees the biased decision policy in the decision to search someone, which then should be denoted as $Z^*=1$ for search and $Z^*=0$ for no search. Though the studies select different variables for the biased decision policy their conceptual frameworks remain comparable.
Again, only on searched people a contraband can be found. The goal is to estimate the possession of a contraband $Y$, but we estimate this from $Y | Z^*=1$.\par

In contrast to \cite{kallus2018}, the authors of this paper argue that the classifier shows the opposite effect; instead of continuing to discriminate the previously disadvantaged group, the classifier exhibits \textit{less} bias as the prejudice against African-Americans increases.\par
As the police becomes more biased towards African-Americans, they search them more leniently. This means that many innocent African-Americans are included in the searched observations. Consequently, the model learns on average lower risk scores for African-Americans. Essentially, the data for African-Americans becomes more noisy, which lowers the predicted probabilities for this group. The authors call this mechanism \textbf{bias reversal}.

\cite{rambachan2016} continue to examine alternative classification tasks that could be constructed from the SQF data. In their second scenario, they train an algorithm to predict whether to search someone in the first place ($Z^* \in \{0, 1\}$). Now the search becomes the target and in this case \textbf{bias inheritance} is observed, meaning the classifier continues to discriminate the historically disadvantaged group.\par
The same happens for a two stage classification task that first predicts whether to search someone and if they searched someone if the individual carries a contraband ($Y \cdot Z^* \in \{0, 1\}$). What happens here is that as more of the stopped African Americans are also searched, the algorithm learns to associate search with more with African Americans than with white people and thus in the future also predicts higher probabilities for a search for African Americans. This is the \textbf{bias inheritance} mechanism.
We can see parallels to this paper to our own case study in the sense that, PoC indeed have lower risk scores (\autoref{fig:fairness_density}, left) and are relatively speaking less often predicted as arrested as white individuals.

As seen in \autoref{tab:sqf_summary} there are more studies that have worked with the SQF data, each with a unique approach to the question of fairness. \cite{Khademi2019FADMELC} are also interested in whether the decision to arrest an individual, after a stop has been made, is discriminatory with respect to race. They design two causal fairness methods, namely the Fair on Average Causal Effect (FACE) and the Fair on Average Causal Effect on the Treated (FACT), to estimate the causal impact of race on the outcome. Their notions of fairness are based on counterfactual reasoning, utilizing Inverse Probability Weighting and Matching to estimate the quantities of interest.
While one of their metrics finds that the odds of being arrested after a stop are higher for Black-Hispanics than for white individuals, the other metric does not show any racial discrimination.\par
\cite{goel2016}, on the other hand, focus on the prediction of the possession of a weapon. They construct a statistical model to account for location specific crime rates and developments of criminal activity over time. After controlling for these factors, the authors conclude that Black and Hispanic individuals are disproportionately often involved in low-risk stops.

% \section*{Bias in, bias out - an alternative perspective}

% mathematical definitions I really need to get the point across
% - population as tupel of random variables (X, U, A, Y)
% - taste-based discriminator https://link.springer.com/referenceworkentry/10.1007/978-981-33-4016-9_1-1

An interesting perspective on this observation can be found in \cite{RambachanBBOEFW}. They take a different stance on the problem of biased training data than \cite{kallus} and question the "bias in, bias out" mechanism.

They formalise the problem as follows. For the decision-maker (the police) an individual is characterised by the random vector $(X, U, A)$, where X and A have the same meaning as in \cite{kallus}
and U is a set of unobserved features. These latent variables are unknown to the algorithm but are characteristics the police bases their decision to stop someone on. In the SQF context this could be the personal impression the officer got of a suspect which is not recorded and hard to measure.

The paper assumes the police is a taste-based classifier against African-Americans. This means they hold some form of prejudice against the group of African-Americans that influences their decision to stop a member of this group.
In general, for stopping any person, an officer incurs a cost c > 0. If they stop an individual that turns out to be involved in criminal activity and is therefore arrested, the officer receives a reward b = 1 \footnote{The reward can set to any number b > 0. We assume b = 1 as in \cite{RambachanBBOEFW} without loss of generality.}. In case of stopping an innocent person b = 0.
For stopping African Americans the payoff an officer expects increases by $\tau > 0$ compared to stopping a white person. The total payoff for stopping an individual is given by:
$$Y + \tau * A - c$$
where $Y$ is the outcome of the stop, $\tau$ is the discrimination parameter, $A \in \{0,1\}$, and $c > 0$ is the cost for stopping a person.
Holding the costs $c$ and the outcome of the stop $Y$ constant, searching an African American results in a higher payoff than searching a white person. The goal of the police is to maximise their payoff. Therefore they stop an individual according to the following threshold rule:
$$Z(X, U, R) = 1(E[Y|X, U, A] \ge c - \tau * A)$$
This means that the threshold for stopping an African American is \textit{lower} than for stopping a white person. Consequently, the police stops African Americans more leniently than white people.
This taste-based discrimination rule is the biased decision policy introduced in \cite{kallus}. In \cite{RambachanBBOEFW} the authors speak of "selective labels" where again the tupel $(Y, X, A, Z)$ is only available for $Z = 1$. 

In short: It actually depends on the outcome and the training sample whether the discrminiation of the previously discriminated (bias interhitance) exists. In some cases it can actuall come to the opposite effect, which they call bias reversal.
The mechanism is as follows: the historically discriminated groups is very represented in the sample as being included is an act of discrimination itself. This means we have more training data for the disadvantaged group, they resemble the target population more, as they were more leniently included, and thus the classifier generalised better to the disadvanteged group.
When we collect more data for the group, we come closer to the target population and our classifier will work better on the target population for the group with more data.


Black people are more leniently stopped, leading to higher stopping rates in for black people in the training data, meaning
more training data for this group. Because we stop black peopel more leniently, we record many innocent black people in our data.
In \cite{kallus} this would lead to a lower learned threshold \footnote{first this leads to lower risk scores for black individuals. And then via fairness adjustments (e.g. for equalized odds) this leads to lower thresholds for black individuals.}
for black individuals. Applied on the target population this would mean that we would predict too many false positive. The threshold estimated from the training 
data is so low that we classify to many people as guilty because in the target populations the scores are actually higher and meet the threshold easily.
In \cite{RambachanBBOEFW} they say that by stopping (searching, they actually talk about searching, not stopping) black people so leniently, our sample for black people comes actually pretty
close to the target population.
In other words, the training data for black people is pretty close to the target data for black people, which means that our classifier will work well on the
target population for black people. \\
To summarise, in \cite{kallus} bias against a group results in a less representative sample. In \cite{RambachanBBOEFW} bias against a group results in a more representative sample.


\textbf{Theorem 1}\\
The prediction for african americans is weakly decreasing in tau. This means, as tau increases (so racial bias increases), the expected value for Y gets actually lower,
so closer to zero, so less often predicted to have a contraband. What is happening? Higher tau means lower searching threshold for african americans.
So the data for african americans becomes "more noisy", more and more innocent people come into our sample, so we predict lower risk for african americans. 
In \cite{RambachanBBOEFW} paper this translates to a more representative training data for african americans and thus also better performance on the general population of african americans.
In \cite{kallus} paper the mechanisms is the same, we also estimate lower risks cores for african americans, but then sth else happens.
I think in Kallus we then do a fairness intervention that leads us to setting a LOWER threshold for african americans, meaning we predict them as
guilty more easily to achieve the same FPR as in the other group. I think in kallus they first formulate it in the strict way, where the police is so biased against african americans
that the stopped african americans are LESS likely to actually have a weapon than the general population. But they relax this setting afterwards.


What happens if we train the logistic classifier (to predict weapon yes no) on the SQF as is (Kallus), don’t do a post processing fairness intervention (NO Hardt et. al)
and test the classifier on the target population (that is created via the weighing method of Kallus and Zhou)? I think according to \cite{RambachanBBOEFW} we should observe bias reversal.

\section{Conclusion}

% Why did our fairness experiment did not show any substantial disparities?\\
% By predicting the arrest of a person we asked a different fairness questions. Fairness in SQF can bee seen as a two-stage problem. The firs stage: was the person stopped? The second stage: what was the result of the stop? For the whole picture it might be better to go in this order. Our tasks jump directly to the second stage. This is not a mistake per se, but one should be aware that these analyses do not reflect the whole process but only a part of it.\\
% On top of this, the mechanisms behind the selection bias in SQF is twisted in the sense that the historically discriminated group is \textit{more present} in the data. Often the situation is that disadvantaged groups form underrepresented minorities, thus the algorithm oversees them and performs worse on them. In the SQF data, however, the algorithm has plenty of observations from PoC to learn from and less from white people.

Certainly the SQF data comes with interesting questions and challenges. We specifically examined fairness and selection bias, but there are more aspects to explore. Historical bias could also play a role and it would be interesting to see how future studies could incorporate this. Moreover, the impact of the class imbalance in the protected attribute on the fairness of the model could be further investigated. The dataset was rightfully recommended by x, offering research possibilities for various disciplines.\\
After a detailed fairness audit, a fairness experiment with various learner and the review of multiple studies our answer to "Is the stop, question, and frisk practice fair?" remains to be: it is complex.\\
With our work we showed that, before any fairness intervention, it is crucial to formulate a concrete fairness question. It is something entirely different to ask if the stop, question, and frisk practice (as a whole) is fair or whether a classifier to predict the arrest of a person trained on the historical stops is fair.\\
The question we formulate can lead to the design of completely different algorithmic tasks and fairness analysis.



\newpage

% \includeonly{chapters/introduction.tex, chapters/results.tex}
% include{}

% \section{Data}
% \label{Data}
% \input{Chapter/Data.tex}
% \newpage
\newpage

% ------------------------------------------------------------------------------
% listoffigures ----------------------------------------------------------------
% ------------------------------------------------------------------------------

% \setcounter{page}{3} % CHANGE

\listoffigures
% \input{Chapter/List of figures.tex}

% ------------------------------------------------------------------------------
% listoftables -----------------------------------------------------------------
% ------------------------------------------------------------------------------

\listoftables
% \input{Chapter/List of tables.tex}


% ------------------------------------------------------------------------------
% APPENDIX ---------------------------------------------------------------------
% ------------------------------------------------------------------------------
    
\pagenumbering{Roman}

\setcounter{page}{5} % CHANGE

\appendix

% \section{Appendix}
% \label{app}
% \input{Chapter/Appendix A.tex}
% \newpage

\section{Electronic Appendix}
\label{el_app}
See the GitHub repository for data, code and illustrations:
\href{https://github.com/juliet-fleischer/SQF_fairness_project}{SQF Fairness Project}
\newpage

% \input{Chapter/Appendix B.tex}
% \newpage

% ------------------------------------------------------------------------------
% BIBLIOGRAPHY -----------------------------------------------------------------
% ------------------------------------------------------------------------------

\RaggedRight
% \bibliography{../literature.bib}
% \bibliographystyle{dcu}
\printbibliography
\newpage
% ------------------------------------------------------------------------------
% DECLARATION OF AUTHORSHIP-----------------------------------------------------
% ------------------------------------------------------------------------------
\Large
\noindent
\textbf{Declaration of authorship} 
\vspace{0.5cm}
\noindent
\normalsize

I hereby declare that the report submitted is my own unaided work. All direct 
or indirect sources used are acknowledged as references. I am aware that the 
Thesis in digital form can be examined for the use of unauthorized aid and in 
order to determine whether the report as a whole or parts incorporated in it may 
be deemed as plagiarism. For the comparison of my work with existing sources I 
agree that it shall be entered in a database where it shall also remain after 
examination, to enable comparison with future Theses submitted. Further rights 
of reproduction and usage, however, are not granted here. This paper was not 
previously presented to another examination board and has not been published.
\\

\vspace{1cm}
\textcolor{orange}{Munich, \mydate} \\

\vspace{3cm}

\noindent\rule{0.5\textwidth}{0.4pt} \\

\textcolor{orange}{\myname}

\end{document}
