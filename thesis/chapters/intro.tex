
% Why should the reader care? fairness is important because ...
The challenge of creating a fair and equitable society has been a concern for society since ancient times. With the rise of artificial intelligence (AI) questions of justice and fairness have taken on new urgency. AI enables automated decision-making systems (ADM) that are now common in law, healthcare, finance, and other fields, where they can affect the lives of people significantly. Despite their ongoing improvements they carry the risk of perpetuating and even exacerbating social injustices.\par

% Whats the issue? the problem is that ...
After a general introduction to the study of fairness in machine learning (fairML), this paper turns its focus to the stop, question, and frisk (SQF) dataset published yearly by the New York Police Department (NYPD). Since 1990 the US Supreme Court has been allowing police officers in New York City to stop individuals if they have a reasonable suspicion that they are involved in criminal activity (Terry v. Ohio (1968), 392 U.S. 1, U.S. Supreme Court).\\
While proponents argue that the stop-and-frisk strategy is an effective crime prevention tool, many criticize the police for disproportionally targetting people of colour (PoC).
In 2013, a federal district court ruled that the way stop-and-frisk was implemented in NYC between 2004 and 2012 was unconstitutional, violating both the Fourth and Fourteenth Amendments.\footnote{\href{https://www.policinginstitute.org/wp-content/uploads/2018/07/5-THINGS-Stop\_Question\_Frisk.pdf}{Link to Report}}
Official statistics show the steep decline in stops after the 2013 judgement. Since then the stops have been kept at a low level.\footnote{\href{https://www.nyclu.org/data/stop-and-frisk-data}{https://www.nyclu.org/data/stop-and-frisk-data}}


% more history, development of stops over time, link to the website


% Why is this paper needed?
The main contribution of this thesis lies in reviewing multiple studies that examine the fairness of SQF from different angles. Though these studies seek to answer the same question—Is stop, question, and frisk fair?—they approach the problem differently and arrive at alternative conclusions.\\
This divergence is not necessarily a contradiction but rather a reflection of the diverse perspectives and objectives that shape fairness research. Each study addresses fairness within its own problem setting, making its conclusions valid within that specific context. However, this can create confusion, as studies with different assumptions and goals may still claim to answer the same overarching question. Our objective is not to identify \textit{the right} approach but to emphasize the importance of understanding data context and problem framing when evaluating fairness.\par
% What will we cover?
The paper is organized in the following way: in Section~\ref{sec:fairness_metrics_methods}, we introduce the most common fairness metrics and techniques used in machine learning.
Next, in Section~\ref{sec:case_study} we apply the theoretical concepts to the real-world SQF dataset. 
The application on real-world data will show difficulties that come with fairness in practice.
This will lead us to explore other studies that have worked with SQF data in Section~\ref{sec:studies}. 

% Important infos about setup
% we focus binary classifiation, binary PA scenario

