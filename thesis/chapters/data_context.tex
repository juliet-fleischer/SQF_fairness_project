How has discrimination been studied so far in the context of SQF in New York City?

Gelman, Fagan, and Kiss (2007): This study analyzed NYPD stop-and-frisk data and found that,
even after controlling for precinct-level crime rates and other variables, Black and Hispanic
individuals were stopped more frequently than White individuals, suggesting that racial bias
could not be fully explained by crime rates or neighborhood demographics.

My first intuition says that like this one might underestimate the racial bias in the following way.
Racial discrimination over a long period of time has lead such unfair circumstances
(lower income, worse education, etc.) that the crime rates are higher in the neighborhoods
where Black and Hispanic individuals live. This means higher crime rates in these neighborhoods
are not the pure result of the individuals' behavior but also the result of the historical bias.
On top of this comes increased policing in these neighborhoods, which further increases the observed
crime rates. 

Stop and Frisk during 2004 to 2012 has been deemed as unconstitutional.