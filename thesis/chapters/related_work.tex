Given the abundance of fairness definitions there has been great efforts to summarise all of them. We do not claim a comprehensive overview of all fairness definitions in our paper but we mainly focus on definitions that become relevant in the case study or are necessary for a nuanced picture on fairness (give the reader some so they know they exist and people think about fairness also in this way).
We refer to \cite{verma2018}, \cite{castelnovo2022} for a good overview; \cite{corbett-davies} and \cite{barocas} for extensive and detailed discussions on fairness definitions and their limitations.
\cite{mehrabi2022} and \cite{caton2024} provide a great overview of fairness methods and sources of bias.
Related to SQF \cite{goel2016} stands out as they combine a socio-ecnonomic approach with advanced statistical methods. With their study, they support the claim that non-white individuals, are over-targeted by the New Yorker police. 

% use their study to back up my own choice for arrrestment as target and little racial disparities
\cite{Badr2022DTFANSP}

% introduce the interesting concept of bias reversal
\cite{RambachanBBOEFW}

% honestly, I want their problem setting and their estimation approach but nothin more
\cite{kallus2018}


% fairness addressed from a causal lense
In \cite{Khademi2019FADMELC} they address fairness from a causal perspective. They specifically measure a form of causal individual fairness for SQF and causal group fairness (FACE, FACT). But also here they come to different fairness statements depending on the metric.

% fairness (in SQF) really depends on the perspective you take! What task do train? What covariates do you take into account?
% How do you want to deploy the algorithm in the future? 