Fairness in machine learning has attracted considerable attention in recent years, leading to a rich literature of definitions and evaluation frameworks. Several works provide broad overviews of these definitions. For example, \cite{verma2018} offers a comprehensive overview of the most popular fairness metrics and \cite{castelnovo2022} highlights their nuances in a compact manner. \cite{corbett-davies} and \cite{barocas} serve as detailed resources that offer deeper insights into common falacies in fairness metrics. Beside the definition of fairness a major area of research is the design of bias migitating techniques to which \cite{mehrabi2022} and \cite{caton2024} provide a great overview. Additionally, the \texttt{mlr3book} serves as an accessible introduction to the practical implementation of fairness metrics.

Beyond these general discussions, a number of studies have focused on the stop, question, and frisk (SQF) dataset—an area where machine learning intersects with socio-economic and statistical analysis. In this context, \cite{gelman2007} has been one of the earlier works that provides both historical context and a sophisticated statistical to show racial disparities in the policing strategy. Building on this, \cite{goel2016} advance the statistical methods further to support the claim that non-white individuals are disproportionately targeted by the New York police.

Moreover, fairness in SQF has also been examined from a causal perspective. For instance, \cite{Khademi2019FADMELC} explores causal individual and group fairness and introduces causal group fairness metrics for SQF. Their study supports the complexity of measuring fairness in SQF practice as their different metrics come to divergent fairness conclusions. The other studies on SQF \cite{Badr2022DTFANSP, RambachanBBOEFW, kallus2018} will be more closely examined in the final chapter of this paper.


% fairness (in SQF) really depends on the perspective you take! What task do train? What covariates do you take into account?
% How do you want to deploy the algorithm in the future? 