They basically argue that i) there is indirect racial discrimination through the precinct (so people in high crime low income areas are generally more likely to be stopped - regardless of race) BUT certain race groups are more likely to be in high crime low income areas.
And ii) on top of this mediated discrimination, police officers are a taste-based discriminator against PoC, stopping them more leniently.

Hit rate: proportion of stops in which the suspect was actually guilty; assessment metric for policing strategy
ex ante probability: probability that the person has a weapon \textit{before} the stop - "the educated guess of the police officer"


When they say they model three ex ante probability of recovering a weapon, they are essentially doing nothing other than taking "possession of a weapon" as target and selecting features that reflect the status of information the police officer had right before stopping a person.