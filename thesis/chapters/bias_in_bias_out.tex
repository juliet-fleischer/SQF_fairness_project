
% In short: It actually depends on the outcome and the training sample whether the discrminiation of the previously discriminated (bias interhitance) exists. In some cases it can actuall come to the opposite effect, which they call bias reversal.
% The mechanism is as follows: the historically discriminated groups is very represented in the sample as being included is an act of discrimination itself. This means we have more training data for the disadvantaged group, they resemble the target population more, as they were more leniently included, and thus the classifier generalised better to the disadvanteged group.
% When we collect more data for the group, we come closer to the target population and our classifier will work better on the target population for the group with more data.


% Black people are more leniently stopped, leading to higher stopping rates in for black people in the training data, meaning
% more training data for this group. Because we stop black peopel more leniently, we record many innocent black people in our data.
% In \cite{kallus2018} this would lead to a lower learned threshold \footnote{first this leads to lower risk scores for black individuals. And then via fairness adjustments (e.g. for equalized odds) this leads to lower thresholds for black individuals.}
% for black individuals. Applied on the target population this would mean that we would predict too many false positive. The threshold estimated from the training 
% data is so low that we classify to many people as guilty because in the target populations the scores are actually higher and meet the threshold easily.
% In \cite{RambachanBBOEFW} they say that by stopping (searching, they actually talk about searching, not stopping) black people so leniently, our sample for black people comes actually pretty
% close to the target population.
% In other words, the training data for black people is pretty close to the target data for black people, which means that our classifier will work well on the
% target population for black people. \\
% To summarise, in \cite{kallus2018} bias against a group results in a less representative sample. In \cite{RambachanBBOEFW} bias against a group results in a more representative sample.


% \textbf{Theorem 1}\\
% The prediction for african americans is weakly decreasing in tau. This means, as tau increases (so racial bias increases), the expected value for Y gets actually lower,
% so closer to zero, so less often predicted to have a contraband. What is happening? Higher tau means lower searching threshold for african americans.
% So the data for african americans becomes "more noisy", more and more innocent people come into our sample, so we predict lower risk for african americans. 
% In \cite{RambachanBBOEFW} paper this translates to a more representative training data for african americans and thus also better performance on the general population of african americans.
% In \cite{kallus2018} paper the mechanisms is the same, we also estimate lower risks cores for african americans, but then sth else happens.
% I think in Kallus we then do a fairness intervention that leads us to setting a LOWER threshold for african americans, meaning we predict them as
% guilty more easily to achieve the same FPR as in the other group. I think in kallus they first formulate it in the strict way, where the police is so biased against african americans
% that the stopped african americans are LESS likely to actually have a weapon than the general population. But they relax this setting afterwards.
